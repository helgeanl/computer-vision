\documentclass[pdftex,12pt,a4paper,twoside]{report}

% ----- Packages -----
                                % Sets document margins
\usepackage[lmargin=25mm,rmargin=25mm,tmargin=27mm,bmargin=30mm]{geometry}

% Language and Document Encoding
\usepackage[english]{babel}     % Applies culturally defined typographical rules.
\usepackage[utf8]{inputenc}     % Adds the ability to accept files of different encodings
\usepackage[T1]{fontenc}        % Allows encoding for fonts to be selected

\selectlanguage{english}

% Referencing & Citations
\usepackage{float}              % Improves floating objects. Lots of packages rely on this being present.
\usepackage{subcaption}         % Adds the ability to add captions to subfigures
\usepackage{cite}               % Basic support for citations (BIBTEX).
\usepackage{caption}            % Allows adding caption to floating objects such as images.

% Graphics packages
\usepackage{graphicx}           % The go-to LaTeX graphics package
\usepackage{wrapfig}            % Allows figures or tables to have text wrapped around them
\usepackage{pgfplots}           % Adds commands for making a range of different plots
\usepackage{rotating}           % Allows rotation of images
\usepackage[section]{placeins}  % Used for limiting images to a particular section

% Tables
\usepackage{booktabs}           % Table formatting intended for scientific writing
\usepackage{tabu}               % Customisable tables
\usepackage{pgfplotstable}      % Additional package for formatting numeric data in tables
\usepackage{longtable}          % Allows the creation of tables spanning multiple pages

% Mathematical Equations
\usepackage{mathtools}          % Enhancing the creation of mathematical equations
\usepackage{amsmath}            % Mathematical facilitation package (amongst others referencing of equations)

% Text and Formatting
\usepackage{verbatim}           % Allows placing text "as-is" into the document, without LaTeX interfering
\usepackage{url}                % Allows insertion of URLs 
\usepackage{color}              % Adds the ability to change text colour.
\usepackage{listings}           % Fallback package in case minted doesn't work.
\usepackage{minted}             % Package for syntax highlighting. NOTE!! Requires python package to be installed
\usepackage{titlesec}           % Allows using the \paragraph command for another level of section subdivision
                                % Adding the \paragraph command to the table of contents. Used code from:
                                % http://tex.stackexchange.com/questions/60209/how-to-add-an-extra-level-of-sections-with-headings-below-subsubsection
\titleformat{\paragraph}{\normalfont\normalsize\bfseries}{\theparagraph}{1em}{}
\titlespacing*{\paragraph}{0pt}{3.25ex plus 1ex minus .2ex}{1.5ex plus .2ex}

% Other
\usepackage{ae}                 % Allows inclusion of fonts into document (improves compatibility)

% (Package) settings configuration
\usepgfplotslibrary{dateplot}

\setlength{\parindent}{0.0in}
\setlength{\parskip}{1.5ex}

\setlength{\fboxsep}{0pt}
\setlength{\fboxrule}{1pt}

\fontfamily{ptm}\selectfont

                                % These two packages must be loaded last, in this order.
\usepackage{hyperref}
\hypersetup{
    unicode=true,
    pdftoolbar=true,
    pdfmenubar=true,
    pdffitwindow=false,
    pdfstartview={FitH},
    pdftitle={Image Processing, IDI, NTNU},
    pdfauthor={All contributing authors},
    pdfsubject={Image Processing, Computer Vision},
    pdfkeywords={ntnu} {image} {processing},
    pdfnewwindow=true,
    colorlinks=false,
    linkcolor=blue,
    citecolor=blue,
    filecolor=blue,
    urlcolor=blue
}
\usepackage[noabbrev]{cleveref} % Used for being able to cite multiple subequations

% ----- Main Document Start -----
% --- Front Page ---

\begin{document}

\thispagestyle{empty}

\begin{center}
    {\Huge\textbf{Image Processing}} \\
    \medskip

    \vspace{1.5cm}

    {\Large TDT4265 Computer Vision}\\

    \vspace{0.6cm}

    {\large Spring 2017}\\

    \vspace{15cm}

    \large{Department of\\Computer and Information Science}

\end{center}

\newpage

% --- Introduction ---

% Configuring counters such that \paragraph commands are visible in the table of contents
\setcounter{tocdepth}{4}
\setcounter{secnumdepth}{4}

\tableofcontents
\clearpage

% --- Chapters ---

% DO NOT MODIFY THIS FILE - ITS CONTENTS HAVE BEEN GENERATED BY A SCRIPT AUTOMATICALLY AND WILL BE OVERWRITTEN WHEN THIS SCRIPT IS EXECUTED AGAIN

\chapter{Introduction, Background and Prerequisites} 
\label{booksection_0_Introduction_Background_and_Prerequisites}
\section{Visual Computing, Image Processing and Computer Vision} 
\label{booksection_1_Visual_Computing_Image_Processing_and_Computer_Vision}
% Insert your contribution in this file.

This section is currently empty.
\section{Example Applications} 
\label{booksection_2_Example_Applications}
\subsection{Autonomy} 
\label{booksection_3_Autonomy}
% Insert your contribution in this file.

This section is currently empty.
\subsection{Medical Image Analysis} 
\label{booksection_4_Medical_Image_Analysis}
% Insert your contribution in this file.

This section is currently empty.
\subsection{Industrial Inspection and Surveillance} 
\label{booksection_5_Industrial_Inspection_and_Surveillance}
% Insert your contribution in this file.

This section is currently empty.
\subsection{Consumer and Game Industry} 
\label{booksection_6_Consumer_and_Game_Industry}
% Insert your contribution in this file.

This section is currently empty.
\section{Human Vision (the Eye and the Brain)} 
\label{booksection_7_Human_Vision_(the_Eye_and_the_Brain)}
% Insert your contribution in this file.

This section is currently empty.
\section{Image Formation and Capturing} 
\label{booksection_8_Image_Formation_and_Capturing}
% Insert your contribution in this file.

This section is currently empty.
\section{Digital Images} 
\label{booksection_9_Digital_Images}
% Insert your contribution in this file.

This section is currently empty.
\section{Features, Machine Learning and Deep Learning} 
\label{booksection_10_Features_Machine_Learning_and_Deep_Learning}
% Insert your contribution in this file.

This section is currently empty.
\section{Programming Prerequisites} 
\label{booksection_11_Programming_Prerequisites}
\subsection{MATLAB} 
\label{booksection_12_MATLAB}
% Insert your contribution in this file.

This section is currently empty.
\subsection{Python} 
\label{booksection_13_Python}
% Insert your contribution in this file.

This section is currently empty.
\section{Mathematical Prerequisites} 
\label{booksection_14_Mathematical_Prerequisites}
% Insert your contribution in this file.

This section is currently empty.
\chapter{Image Processing Fundamentals} 
\label{booksection_15_Image_Processing_Fundamentals}
\section{Image Enhancement in the Spatial Domain} 
\label{booksection_16_Image_Enhancement_in_the_Spatial_Domain}
\subsection{Point Processing and Intensity Transformations} 
\label{booksection_17_Point_Processing_and_Intensity_Transformations}
\subsubsection{Identity and Inverse Transformations, Basic Concepts} 
\label{booksection_18_Identity_and_Inverse_Transformations_Basic_Concepts}
% Insert your contribution in this file.

This section is currently empty.
\subsubsection{Logarithmic Transform and its Inverse} 
\label{booksection_19_Logarithmic_Transform_and_its_Inverse}
% Insert your contribution in this file.

This section is currently empty.
\subsubsection{Gamma Transform} 
\label{booksection_20_Gamma_Transform}
% Insert your contribution in this file.

This section is currently empty.
\subsection{Histogram-based methods} 
\label{booksection_21_Histogram-based_methods}
\subsubsection{Histogram, Normalized Histogram and Probability, Cumulative Histogram} 
\label{booksection_22_Histogram_Normalized_Histogram_and_Probability_Cumulative_Histogram}
% Insert your contribution in this file.

This section is currently empty.
\subsubsection{Histogram Equalization} 
\label{booksection_23_Histogram_Equalization}
% Insert your contribution in this file.

This section is currently empty.
\subsubsection{Local and Adaptive Methods} 
\label{booksection_24_Local_and_Adaptive_Methods}
% Insert your contribution in this file.

This section is currently empty.
\subsection{Neighborhood Filtering} 
\label{booksection_25_Neighborhood_Filtering}
\subsubsection{Basic Concepts} 
\label{booksection_26_Basic_Concepts}
\paragraph{Convolution Kernels, Filters, and Masks} 
\label{booksection_27_Convolution_Kernels_Filters_and_Masks}
% Insert your contribution in this file.

This section is currently empty.
\paragraph{Linearity property and Filtering Impulse Images} 
\label{booksection_28_Linearity_property_and_Filtering_Impulse_Images}
% Insert your contribution in this file.

This section is currently empty.
\paragraph{Correlation versus Convolution and Symmetric filters} 
\label{booksection_29_Correlation_versus_Convolution_and_Symmetric_filters}
% Insert your contribution in this file.

This section is currently empty.
\paragraph{Properties of Convolution Kernels and Separability} 
\label{booksection_30_Properties_of_Convolution_Kernels_and_Separability}
% Insert your contribution in this file.

This section is currently empty.
\paragraph{Boundary Issues and Padding} 
\label{booksection_31_Boundary_Issues_and_Padding}
% Insert your contribution in this file.

This section is currently empty.
\subsubsection{Image Smoothing, Noise and Template Matching} 
\label{booksection_32_Image_Smoothing_Noise_and_Template_Matching}
\paragraph{Box Filters, Normalization and Noise} 
\label{booksection_33_Box_Filters_Normalization_and_Noise}
% Insert your contribution in this file.

This section is currently empty.
\paragraph{Gaussian Filters, Gaussian Approximations and Sigmas} 
\label{booksection_34_Gaussian_Filters_Gaussian_Approximations_and_Sigmas}
% Insert your contribution in this file.

This section is currently empty.
\paragraph{Non-linear Filters} 
\label{booksection_35_Non-linear_Filters}
% Insert your contribution in this file.

This section is currently empty.
\paragraph{Template Matching} 
\label{booksection_36_Template_Matching}
% Insert your contribution in this file.

This section is currently empty.
\subsubsection{Image Sharpening and Gradient Filters} 
\label{booksection_37_Image_Sharpening_and_Gradient_Filters}
\paragraph{Basic Derivative Filters} 
\label{booksection_38_Basic_Derivative_Filters}
% Insert your contribution in this file.

This section is currently empty.
\paragraph{First and Second Derivatives} 
\label{booksection_39_First_and_Second_Derivatives}
% Insert your contribution in this file.

This section is currently empty.
\paragraph{Derivatives of Gaussian. Sigma and Scale} 
\label{booksection_40_Derivatives_of_Gaussian_Sigma_and_Scale}
% Insert your contribution in this file.

This section is currently empty.
\section{Image Enhancement in the Frequency Domain} 
\label{booksection_41_Image_Enhancement_in_the_Frequency_Domain}
\subsection{Basic Concepts} 
\label{booksection_42_Basic_Concepts}
\subsubsection{Intro and Motivation} 
\label{booksection_43_Intro_and_Motivation}
% Insert your contribution in this file.

This section is currently empty.
\subsubsection{Various Key Terms} 
\label{booksection_44_Various_Key_Terms}
% Insert your contribution in this file.

This section is currently empty.
\subsubsection{Decomposition, Fourier Basis, and Their Interpretation} 
\label{booksection_45_Decomposition_Fourier_Basis_and_Their_Interpretation}
% Insert your contribution in this file.

This section is currently empty.
\subsubsection{The Forward Fourier Transform and its Inverse} 
\label{booksection_46_The_Forward_Fourier_Transform_and_its_Inverse}
% Insert your contribution in this file.

This section is currently empty.
\subsubsection{Important pairs and properties} 
\label{booksection_47_Important_pairs_and_properties}
% Insert your contribution in this file.

This section is currently empty.
\subsubsection{Convolution theorem} 
\label{booksection_48_Convolution_theorem}
% Insert your contribution in this file.

This section is currently empty.
\subsection{Filtering in the Frequency Domain} 
\label{booksection_49_Filtering_in_the_Frequency_Domain}
\subsubsection{Illustrated Explanation of the Frequency Domain Filtering Process} 
\label{booksection_50_Illustrated_Explanation_of_the_Frequency_Domain_Filtering_Process}
% Insert your contribution in this file.

This section is currently empty.
\subsubsection{Common Frequency Domain Filters} 
\label{booksection_51_Common_Frequency_Domain_Filters}
% Insert your contribution in this file.

This section is currently empty.
\subsubsection{Ringing} 
\label{booksection_52_Ringing}
% Insert your contribution in this file.

This section is currently empty.
\subsubsection{Practical Issues} 
\label{booksection_53_Practical_Issues}
% Insert your contribution in this file.

This section is currently empty.
\subsection{Sampling, Reconstruction, and Aliasing} 
\label{booksection_54_Sampling_Reconstruction_and_Aliasing}
\subsubsection{Introduction into Sampling, Reconstruction and Aliasing} 
\label{booksection_55_Introduction_into_Sampling_Reconstruction_and_Aliasing}
% Insert your contribution in this file.

This section is currently empty.
\subsubsection{Impulse train, Sampling, Oversampling, and the Sampling Theorem} 
\label{booksection_56_Impulse_train_Sampling_Oversampling_and_the_Sampling_Theorem}
% Insert your contribution in this file.

This section is currently empty.
\subsubsection{Aliasing and Anti-aliasing} 
\label{booksection_57_Aliasing_and_Anti-aliasing}
% Insert your contribution in this file.

This section is currently empty.
\section{Image Segmentation} 
\label{booksection_58_Image_Segmentation}
\subsection{Basic Concepts} 
\label{booksection_59_Basic_Concepts}
\subsubsection{Introduction Into Basic Concepts} 
\label{booksection_60_Introduction_Into_Basic_Concepts}
% Insert your contribution in this file.

This section is currently empty.
\subsection{Boundary Based Methods and the Hough Transform } 
\label{booksection_61_Boundary_Based_Methods_and_the_Hough_Transform_}
\subsubsection{Edges versus Boundaries} 
\label{booksection_62_Edges_versus_Boundaries}
% Insert your contribution in this file.

This section is currently empty.
\subsubsection{Canny Edge Detector} 
\label{booksection_63_Canny_Edge_Detector}
% Insert your contribution in this file.

This section is currently empty.
\subsubsection{Hough Transform } 
\label{booksection_64_Hough_Transform_}
\paragraph{Hough Transform for Locating Lines} 
\label{booksection_65_Hough_Transform_for_Locating_Lines}
% Insert your contribution in this file.

This section is currently empty.
\paragraph{Hough Transform for Locating Circles} 
\label{booksection_66_Hough_Transform_for_Locating_Circles}
% Insert your contribution in this file.

This section is currently empty.
\paragraph{Generalized Hough Transform } 
\label{booksection_67_Generalized_Hough_Transform_}
% Insert your contribution in this file.

This section is currently empty.
\subsection{Region Based Methods} 
\label{booksection_68_Region_Based_Methods}
\subsubsection{Thresholding} 
\label{booksection_69_Thresholding}
\paragraph{Basic Global Thresholding} 
\label{booksection_70_Basic_Global_Thresholding}
% Insert your contribution in this file.

This section is currently empty.
\paragraph{Optimal Global Thresholding and Otsu's method} 
\label{booksection_71_Optimal_Global_Thresholding_and_Otsus_method}
% Insert your contribution in this file.

This section is currently empty.
\paragraph{Multiple, and Local Thresholds} 
\label{booksection_72_Multiple_and_Local_Thresholds}
% Insert your contribution in this file.

This section is currently empty.
\subsubsection{Split-and-Merge Segmentation} 
\label{booksection_73_Split-and-Merge_Segmentation}
% Insert your contribution in this file.

This section is currently empty.
\subsubsection{Region Growing} 
\label{booksection_74_Region_Growing}
% Insert your contribution in this file.

This section is currently empty.
\section{Mathematical Morphology} 
\label{booksection_75_Mathematical_Morphology}
\subsection{Basic Concepts} 
\label{booksection_76_Basic_Concepts}
\subsubsection{Binary Images and Set Operations} 
\label{booksection_77_Binary_Images_and_Set_Operations}
% Insert your contribution in this file.

This section is currently empty.
\subsubsection{The Structuring Element} 
\label{booksection_78_The_Structuring_Element}
% Insert your contribution in this file.

This section is currently empty.
\subsection{Basic Morphological Operations} 
\label{booksection_79_Basic_Morphological_Operations}
\subsubsection{Erosion} 
\label{booksection_80_Erosion}
% Insert your contribution in this file.

This section is currently empty.
\subsubsection{Dilation} 
\label{booksection_81_Dilation}
% Insert your contribution in this file.

This section is currently empty.
\subsubsection{Opening} 
\label{booksection_82_Opening}
% Insert your contribution in this file.

This section is currently empty.
\subsubsection{Closing} 
\label{booksection_83_Closing}
% Insert your contribution in this file.

This section is currently empty.
\subsection{Morphological Algorithms and Methods} 
\label{booksection_84_Morphological_Algorithms_and_Methods}
\subsubsection{Boundary Extraction} 
\label{booksection_85_Boundary_Extraction}
% Insert your contribution in this file.

This section is currently empty.
\subsubsection{Connected Components} 
\label{booksection_86_Connected_Components}
% Insert your contribution in this file.

This section is currently empty.
\subsubsection{Region filling} 
\label{booksection_87_Region_filling}
% Insert your contribution in this file.

This section is currently empty.
\subsubsection{Morphological Reconstruction} 
\label{booksection_88_Morphological_Reconstruction}
% Insert your contribution in this file.

This section is currently empty.
\subsubsection{Labelling Connected Components} 
\label{booksection_89_Labelling_Connected_Components}
% Insert your contribution in this file.

This section is currently empty.
\subsubsection{Hit-and-Miss Transform} 
\label{booksection_90_Hit-and-Miss_Transform}
% Insert your contribution in this file.

This section is currently empty.
\subsubsection{Thinning and Skeletonization} 
\label{booksection_91_Thinning_and_Skeletonization}
% Insert your contribution in this file.

This section is currently empty.
\subsubsection{Distance Transform} 
\label{booksection_92_Distance_Transform}
% Insert your contribution in this file.

This section is currently empty.
\subsubsection{Watershed Segmentation} 
\label{booksection_93_Watershed_Segmentation}
% Insert your contribution in this file.

This section is currently empty.
\subsection{Grayscale Morphology} 
\label{booksection_94_Grayscale_Morphology}
\subsubsection{Morphology with Grayscale Images} 
\label{booksection_95_Morphology_with_Grayscale_Images}
% Insert your contribution in this file.

This section is currently empty.
\subsubsection{Erosion} 
\label{booksection_96_Erosion}
% Insert your contribution in this file.

This section is currently empty.
\subsubsection{Dilation} 
\label{booksection_97_Dilation}
% Insert your contribution in this file.

This section is currently empty.
\subsubsection{Opening} 
\label{booksection_98_Opening}
% Insert your contribution in this file.

This section is currently empty.
\subsubsection{Closing} 
\label{booksection_99_Closing}
% Insert your contribution in this file.

This section is currently empty.
\subsubsection{Top Hat Transform} 
\label{booksection_100_Top_Hat_Transform}
% Insert your contribution in this file.

This section is currently empty.


% --- Appendices ---

\appendix
% DO NOT MODIFY THIS FILE - ITS CONTENTS HAVE BEEN GENERATED BY A SCRIPT AUTOMATICALLY AND WILL BE OVERWRITTEN WHEN THIS SCRIPT IS EXECUTED AGAIN

% If you have something which is relevant but too long to put in the main document, you can create an appendix in this file.
% Make sure to add a "\chapter{}" command to allow the appendix to be listed and referenced.
% You can otherwise leave this document empty.
% If you have something which is relevant but too long to put in the main document, you can create an appendix in this file.
% Make sure to add a "\chapter{}" command to allow the appendix to be listed and referenced.
% You can otherwise leave this document empty.
% If you have something which is relevant but too long to put in the main document, you can create an appendix in this file.
% Make sure to add a "\chapter{}" command to allow the appendix to be listed and referenced.
% You can otherwise leave this document empty.
% If you have something which is relevant but too long to put in the main document, you can create an appendix in this file.
% Make sure to add a "\chapter{}" command to allow the appendix to be listed and referenced.
% You can otherwise leave this document empty.
% If you have something which is relevant but too long to put in the main document, you can create an appendix in this file.
% Make sure to add a "\chapter{}" command to allow the appendix to be listed and referenced.
% You can otherwise leave this document empty.
% If you have something which is relevant but too long to put in the main document, you can create an appendix in this file.
% Make sure to add a "\chapter{}" command to allow the appendix to be listed and referenced.
% You can otherwise leave this document empty.
% If you have something which is relevant but too long to put in the main document, you can create an appendix in this file.
% Make sure to add a "\chapter{}" command to allow the appendix to be listed and referenced.
% You can otherwise leave this document empty.
% If you have something which is relevant but too long to put in the main document, you can create an appendix in this file.
% Make sure to add a "\chapter{}" command to allow the appendix to be listed and referenced.
% You can otherwise leave this document empty.
% If you have something which is relevant but too long to put in the main document, you can create an appendix in this file.
% Make sure to add a "\chapter{}" command to allow the appendix to be listed and referenced.
% You can otherwise leave this document empty.
% If you have something which is relevant but too long to put in the main document, you can create an appendix in this file.
% Make sure to add a "\chapter{}" command to allow the appendix to be listed and referenced.
% You can otherwise leave this document empty.
% If you have something which is relevant but too long to put in the main document, you can create an appendix in this file.
% Make sure to add a "\chapter{}" command to allow the appendix to be listed and referenced.
% You can otherwise leave this document empty.
% If you have something which is relevant but too long to put in the main document, you can create an appendix in this file.
% Make sure to add a "\chapter{}" command to allow the appendix to be listed and referenced.
% You can otherwise leave this document empty.
% If you have something which is relevant but too long to put in the main document, you can create an appendix in this file.
% Make sure to add a "\chapter{}" command to allow the appendix to be listed and referenced.
% You can otherwise leave this document empty.
% If you have something which is relevant but too long to put in the main document, you can create an appendix in this file.
% Make sure to add a "\chapter{}" command to allow the appendix to be listed and referenced.
% You can otherwise leave this document empty.
% If you have something which is relevant but too long to put in the main document, you can create an appendix in this file.
% Make sure to add a "\chapter{}" command to allow the appendix to be listed and referenced.
% You can otherwise leave this document empty.
% If you have something which is relevant but too long to put in the main document, you can create an appendix in this file.
% Make sure to add a "\chapter{}" command to allow the appendix to be listed and referenced.
% You can otherwise leave this document empty.
% If you have something which is relevant but too long to put in the main document, you can create an appendix in this file.
% Make sure to add a "\chapter{}" command to allow the appendix to be listed and referenced.
% You can otherwise leave this document empty.
% If you have something which is relevant but too long to put in the main document, you can create an appendix in this file.
% Make sure to add a "\chapter{}" command to allow the appendix to be listed and referenced.
% You can otherwise leave this document empty.
% If you have something which is relevant but too long to put in the main document, you can create an appendix in this file.
% Make sure to add a "\chapter{}" command to allow the appendix to be listed and referenced.
% You can otherwise leave this document empty.
% If you have something which is relevant but too long to put in the main document, you can create an appendix in this file.
% Make sure to add a "\chapter{}" command to allow the appendix to be listed and referenced.
% You can otherwise leave this document empty.
% If you have something which is relevant but too long to put in the main document, you can create an appendix in this file.
% Make sure to add a "\chapter{}" command to allow the appendix to be listed and referenced.
% You can otherwise leave this document empty.
% If you have something which is relevant but too long to put in the main document, you can create an appendix in this file.
% Make sure to add a "\chapter{}" command to allow the appendix to be listed and referenced.
% You can otherwise leave this document empty.
% If you have something which is relevant but too long to put in the main document, you can create an appendix in this file.
% Make sure to add a "\chapter{}" command to allow the appendix to be listed and referenced.
% You can otherwise leave this document empty.
% If you have something which is relevant but too long to put in the main document, you can create an appendix in this file.
% Make sure to add a "\chapter{}" command to allow the appendix to be listed and referenced.
% You can otherwise leave this document empty.
% If you have something which is relevant but too long to put in the main document, you can create an appendix in this file.
% Make sure to add a "\chapter{}" command to allow the appendix to be listed and referenced.
% You can otherwise leave this document empty.
% If you have something which is relevant but too long to put in the main document, you can create an appendix in this file.
% Make sure to add a "\chapter{}" command to allow the appendix to be listed and referenced.
% You can otherwise leave this document empty.
% If you have something which is relevant but too long to put in the main document, you can create an appendix in this file.
% Make sure to add a "\chapter{}" command to allow the appendix to be listed and referenced.
% You can otherwise leave this document empty.
% If you have something which is relevant but too long to put in the main document, you can create an appendix in this file.
% Make sure to add a "\chapter{}" command to allow the appendix to be listed and referenced.
% You can otherwise leave this document empty.
% If you have something which is relevant but too long to put in the main document, you can create an appendix in this file.
% Make sure to add a "\chapter{}" command to allow the appendix to be listed and referenced.
% You can otherwise leave this document empty.
% If you have something which is relevant but too long to put in the main document, you can create an appendix in this file.
% Make sure to add a "\chapter{}" command to allow the appendix to be listed and referenced.
% You can otherwise leave this document empty.
% If you have something which is relevant but too long to put in the main document, you can create an appendix in this file.
% Make sure to add a "\chapter{}" command to allow the appendix to be listed and referenced.
% You can otherwise leave this document empty.
% If you have something which is relevant but too long to put in the main document, you can create an appendix in this file.
% Make sure to add a "\chapter{}" command to allow the appendix to be listed and referenced.
% You can otherwise leave this document empty.
% If you have something which is relevant but too long to put in the main document, you can create an appendix in this file.
% Make sure to add a "\chapter{}" command to allow the appendix to be listed and referenced.
% You can otherwise leave this document empty.
% If you have something which is relevant but too long to put in the main document, you can create an appendix in this file.
% Make sure to add a "\chapter{}" command to allow the appendix to be listed and referenced.
% You can otherwise leave this document empty.
% If you have something which is relevant but too long to put in the main document, you can create an appendix in this file.
% Make sure to add a "\chapter{}" command to allow the appendix to be listed and referenced.
% You can otherwise leave this document empty.
% If you have something which is relevant but too long to put in the main document, you can create an appendix in this file.
% Make sure to add a "\chapter{}" command to allow the appendix to be listed and referenced.
% You can otherwise leave this document empty.
% If you have something which is relevant but too long to put in the main document, you can create an appendix in this file.
% Make sure to add a "\chapter{}" command to allow the appendix to be listed and referenced.
% You can otherwise leave this document empty.
% If you have something which is relevant but too long to put in the main document, you can create an appendix in this file.
% Make sure to add a "\chapter{}" command to allow the appendix to be listed and referenced.
% You can otherwise leave this document empty.
% If you have something which is relevant but too long to put in the main document, you can create an appendix in this file.
% Make sure to add a "\chapter{}" command to allow the appendix to be listed and referenced.
% You can otherwise leave this document empty.
% If you have something which is relevant but too long to put in the main document, you can create an appendix in this file.
% Make sure to add a "\chapter{}" command to allow the appendix to be listed and referenced.
% You can otherwise leave this document empty.
% If you have something which is relevant but too long to put in the main document, you can create an appendix in this file.
% Make sure to add a "\chapter{}" command to allow the appendix to be listed and referenced.
% You can otherwise leave this document empty.
% If you have something which is relevant but too long to put in the main document, you can create an appendix in this file.
% Make sure to add a "\chapter{}" command to allow the appendix to be listed and referenced.
% You can otherwise leave this document empty.
% If you have something which is relevant but too long to put in the main document, you can create an appendix in this file.
% Make sure to add a "\chapter{}" command to allow the appendix to be listed and referenced.
% You can otherwise leave this document empty.
% If you have something which is relevant but too long to put in the main document, you can create an appendix in this file.
% Make sure to add a "\chapter{}" command to allow the appendix to be listed and referenced.
% You can otherwise leave this document empty.
% If you have something which is relevant but too long to put in the main document, you can create an appendix in this file.
% Make sure to add a "\chapter{}" command to allow the appendix to be listed and referenced.
% You can otherwise leave this document empty.
% If you have something which is relevant but too long to put in the main document, you can create an appendix in this file.
% Make sure to add a "\chapter{}" command to allow the appendix to be listed and referenced.
% You can otherwise leave this document empty.
% If you have something which is relevant but too long to put in the main document, you can create an appendix in this file.
% Make sure to add a "\chapter{}" command to allow the appendix to be listed and referenced.
% You can otherwise leave this document empty.
% If you have something which is relevant but too long to put in the main document, you can create an appendix in this file.
% Make sure to add a "\chapter{}" command to allow the appendix to be listed and referenced.
% You can otherwise leave this document empty.
% If you have something which is relevant but too long to put in the main document, you can create an appendix in this file.
% Make sure to add a "\chapter{}" command to allow the appendix to be listed and referenced.
% You can otherwise leave this document empty.
% If you have something which is relevant but too long to put in the main document, you can create an appendix in this file.
% Make sure to add a "\chapter{}" command to allow the appendix to be listed and referenced.
% You can otherwise leave this document empty.
% If you have something which is relevant but too long to put in the main document, you can create an appendix in this file.
% Make sure to add a "\chapter{}" command to allow the appendix to be listed and referenced.
% You can otherwise leave this document empty.
% If you have something which is relevant but too long to put in the main document, you can create an appendix in this file.
% Make sure to add a "\chapter{}" command to allow the appendix to be listed and referenced.
% You can otherwise leave this document empty.
% If you have something which is relevant but too long to put in the main document, you can create an appendix in this file.
% Make sure to add a "\chapter{}" command to allow the appendix to be listed and referenced.
% You can otherwise leave this document empty.
% If you have something which is relevant but too long to put in the main document, you can create an appendix in this file.
% Make sure to add a "\chapter{}" command to allow the appendix to be listed and referenced.
% You can otherwise leave this document empty.
% If you have something which is relevant but too long to put in the main document, you can create an appendix in this file.
% Make sure to add a "\chapter{}" command to allow the appendix to be listed and referenced.
% You can otherwise leave this document empty.
% If you have something which is relevant but too long to put in the main document, you can create an appendix in this file.
% Make sure to add a "\chapter{}" command to allow the appendix to be listed and referenced.
% You can otherwise leave this document empty.
% If you have something which is relevant but too long to put in the main document, you can create an appendix in this file.
% Make sure to add a "\chapter{}" command to allow the appendix to be listed and referenced.
% You can otherwise leave this document empty.
% If you have something which is relevant but too long to put in the main document, you can create an appendix in this file.
% Make sure to add a "\chapter{}" command to allow the appendix to be listed and referenced.
% You can otherwise leave this document empty.
% If you have something which is relevant but too long to put in the main document, you can create an appendix in this file.
% Make sure to add a "\chapter{}" command to allow the appendix to be listed and referenced.
% You can otherwise leave this document empty.
% If you have something which is relevant but too long to put in the main document, you can create an appendix in this file.
% Make sure to add a "\chapter{}" command to allow the appendix to be listed and referenced.
% You can otherwise leave this document empty.
% If you have something which is relevant but too long to put in the main document, you can create an appendix in this file.
% Make sure to add a "\chapter{}" command to allow the appendix to be listed and referenced.
% You can otherwise leave this document empty.
% If you have something which is relevant but too long to put in the main document, you can create an appendix in this file.
% Make sure to add a "\chapter{}" command to allow the appendix to be listed and referenced.
% You can otherwise leave this document empty.
% If you have something which is relevant but too long to put in the main document, you can create an appendix in this file.
% Make sure to add a "\chapter{}" command to allow the appendix to be listed and referenced.
% You can otherwise leave this document empty.
% If you have something which is relevant but too long to put in the main document, you can create an appendix in this file.
% Make sure to add a "\chapter{}" command to allow the appendix to be listed and referenced.
% You can otherwise leave this document empty.
% If you have something which is relevant but too long to put in the main document, you can create an appendix in this file.
% Make sure to add a "\chapter{}" command to allow the appendix to be listed and referenced.
% You can otherwise leave this document empty.
% If you have something which is relevant but too long to put in the main document, you can create an appendix in this file.
% Make sure to add a "\chapter{}" command to allow the appendix to be listed and referenced.
% You can otherwise leave this document empty.
% If you have something which is relevant but too long to put in the main document, you can create an appendix in this file.
% Make sure to add a "\chapter{}" command to allow the appendix to be listed and referenced.
% You can otherwise leave this document empty.
% If you have something which is relevant but too long to put in the main document, you can create an appendix in this file.
% Make sure to add a "\chapter{}" command to allow the appendix to be listed and referenced.
% You can otherwise leave this document empty.
% If you have something which is relevant but too long to put in the main document, you can create an appendix in this file.
% Make sure to add a "\chapter{}" command to allow the appendix to be listed and referenced.
% You can otherwise leave this document empty.
% If you have something which is relevant but too long to put in the main document, you can create an appendix in this file.
% Make sure to add a "\chapter{}" command to allow the appendix to be listed and referenced.
% You can otherwise leave this document empty.
% If you have something which is relevant but too long to put in the main document, you can create an appendix in this file.
% Make sure to add a "\chapter{}" command to allow the appendix to be listed and referenced.
% You can otherwise leave this document empty.
% If you have something which is relevant but too long to put in the main document, you can create an appendix in this file.
% Make sure to add a "\chapter{}" command to allow the appendix to be listed and referenced.
% You can otherwise leave this document empty.
% If you have something which is relevant but too long to put in the main document, you can create an appendix in this file.
% Make sure to add a "\chapter{}" command to allow the appendix to be listed and referenced.
% You can otherwise leave this document empty.
% If you have something which is relevant but too long to put in the main document, you can create an appendix in this file.
% Make sure to add a "\chapter{}" command to allow the appendix to be listed and referenced.
% You can otherwise leave this document empty.
% If you have something which is relevant but too long to put in the main document, you can create an appendix in this file.
% Make sure to add a "\chapter{}" command to allow the appendix to be listed and referenced.
% You can otherwise leave this document empty.


% --- Document Bibliography ---

\clearpage
\addcontentsline{toc}{chapter}{Bibliography}
\bibliographystyle{ieeetr}
% DO NOT MODIFY THIS FILE - ITS CONTENTS HAVE BEEN GENERATED BY A SCRIPT AUTOMATICALLY AND WILL BE OVERWRITTEN WHEN THIS SCRIPT IS EXECUTED AGAIN

\bibliography{chapters/01_Introduction_Background_and_Prerequisites/01_Visual_Computing_Image/references,chapters/01_Introduction_Background_and_Prerequisites/02_Example_Applications/01_Autonomy/references,chapters/01_Introduction_Background_and_Prerequisites/02_Example_Applications/02_Medical_Image_Analysis/references,chapters/01_Introduction_Background_and_Prerequisites/02_Example_Applications/03_Industrial_Inspection_/references,chapters/01_Introduction_Background_and_Prerequisites/02_Example_Applications/04_Consumer_and_Game_Indu/references,chapters/01_Introduction_Background_and_Prerequisites/03_Human_Vision_(the_Eye_/references,chapters/01_Introduction_Background_and_Prerequisites/04_Image_Formation_and_Ca/references,chapters/01_Introduction_Background_and_Prerequisites/05_Digital_Images/references,chapters/01_Introduction_Background_and_Prerequisites/06_Features_Machine_Learn/references,chapters/01_Introduction_Background_and_Prerequisites/07_Programming_Prerequisites/01_MATLAB/references,chapters/01_Introduction_Background_and_Prerequisites/07_Programming_Prerequisites/02_Python/references,chapters/01_Introduction_Background_and_Prerequisites/08_Mathematical_Prerequis/references,chapters/02_Image_Processing_Fundamentals/01_Image_Enhancement_in_the_Spatial_Domain/01_Point_Processing_and_Intensity_Transformations/01_Identity_and_Inverse_T/references,chapters/02_Image_Processing_Fundamentals/01_Image_Enhancement_in_the_Spatial_Domain/01_Point_Processing_and_Intensity_Transformations/02_Logarithmic_Transform_/references,chapters/02_Image_Processing_Fundamentals/01_Image_Enhancement_in_the_Spatial_Domain/01_Point_Processing_and_Intensity_Transformations/03_Gamma_Transform/references,chapters/02_Image_Processing_Fundamentals/01_Image_Enhancement_in_the_Spatial_Domain/02_Histogram-based_methods/01_Histogram_Normalized_H/references,chapters/02_Image_Processing_Fundamentals/01_Image_Enhancement_in_the_Spatial_Domain/02_Histogram-based_methods/02_Histogram_Equalization/references,chapters/02_Image_Processing_Fundamentals/01_Image_Enhancement_in_the_Spatial_Domain/02_Histogram-based_methods/03_Local_and_Adaptive_Met/references,chapters/02_Image_Processing_Fundamentals/01_Image_Enhancement_in_the_Spatial_Domain/03_Neighborhood_Filtering/01_Basic_Concepts/01_Convolution_Kernels_Fi/references,chapters/02_Image_Processing_Fundamentals/01_Image_Enhancement_in_the_Spatial_Domain/03_Neighborhood_Filtering/01_Basic_Concepts/02_Linearity_property_and/references,chapters/02_Image_Processing_Fundamentals/01_Image_Enhancement_in_the_Spatial_Domain/03_Neighborhood_Filtering/01_Basic_Concepts/03_Correlation_versus_Con/references,chapters/02_Image_Processing_Fundamentals/01_Image_Enhancement_in_the_Spatial_Domain/03_Neighborhood_Filtering/01_Basic_Concepts/04_Properties_of_Convolut/references,chapters/02_Image_Processing_Fundamentals/01_Image_Enhancement_in_the_Spatial_Domain/03_Neighborhood_Filtering/01_Basic_Concepts/05_Boundary_Issues_and_Pa/references,chapters/02_Image_Processing_Fundamentals/01_Image_Enhancement_in_the_Spatial_Domain/03_Neighborhood_Filtering/02_Image_Smoothing_Noise_and_Template_Matching/01_Box_Filters_Normalizat/references,chapters/02_Image_Processing_Fundamentals/01_Image_Enhancement_in_the_Spatial_Domain/03_Neighborhood_Filtering/02_Image_Smoothing_Noise_and_Template_Matching/02_Gaussian_Filters_Gauss/references,chapters/02_Image_Processing_Fundamentals/01_Image_Enhancement_in_the_Spatial_Domain/03_Neighborhood_Filtering/02_Image_Smoothing_Noise_and_Template_Matching/03_Non-linear_Filters/references,chapters/02_Image_Processing_Fundamentals/01_Image_Enhancement_in_the_Spatial_Domain/03_Neighborhood_Filtering/02_Image_Smoothing_Noise_and_Template_Matching/04_Template_Matching/references,chapters/02_Image_Processing_Fundamentals/01_Image_Enhancement_in_the_Spatial_Domain/03_Neighborhood_Filtering/03_Image_Sharpening_and_Gradient_Filters/01_Basic_Derivative_Filte/references,chapters/02_Image_Processing_Fundamentals/01_Image_Enhancement_in_the_Spatial_Domain/03_Neighborhood_Filtering/03_Image_Sharpening_and_Gradient_Filters/02_First_and_Second_Deriv/references,chapters/02_Image_Processing_Fundamentals/01_Image_Enhancement_in_the_Spatial_Domain/03_Neighborhood_Filtering/03_Image_Sharpening_and_Gradient_Filters/03_Derivatives_of_Gaussia/references,chapters/02_Image_Processing_Fundamentals/02_Image_Enhancement_in_the_Frequency_Domain/01_Basic_Concepts/01_Intro_and_Motivation/references,chapters/02_Image_Processing_Fundamentals/02_Image_Enhancement_in_the_Frequency_Domain/01_Basic_Concepts/02_Various_Key_Terms/references,chapters/02_Image_Processing_Fundamentals/02_Image_Enhancement_in_the_Frequency_Domain/01_Basic_Concepts/03_Decomposition_Fourier_/references,chapters/02_Image_Processing_Fundamentals/02_Image_Enhancement_in_the_Frequency_Domain/01_Basic_Concepts/04_The_Forward_Fourier_Tr/references,chapters/02_Image_Processing_Fundamentals/02_Image_Enhancement_in_the_Frequency_Domain/01_Basic_Concepts/05_Important_pairs_and_pr/references,chapters/02_Image_Processing_Fundamentals/02_Image_Enhancement_in_the_Frequency_Domain/01_Basic_Concepts/06_Convolution_theorem/references,chapters/02_Image_Processing_Fundamentals/02_Image_Enhancement_in_the_Frequency_Domain/02_Filtering_in_the_Frequency_Domain/01_Illustrated_Explanatio/references,chapters/02_Image_Processing_Fundamentals/02_Image_Enhancement_in_the_Frequency_Domain/02_Filtering_in_the_Frequency_Domain/02_Common_Frequency_Domai/references,chapters/02_Image_Processing_Fundamentals/02_Image_Enhancement_in_the_Frequency_Domain/02_Filtering_in_the_Frequency_Domain/03_Ringing/references,chapters/02_Image_Processing_Fundamentals/02_Image_Enhancement_in_the_Frequency_Domain/02_Filtering_in_the_Frequency_Domain/04_Practical_Issues/references,chapters/02_Image_Processing_Fundamentals/02_Image_Enhancement_in_the_Frequency_Domain/03_Sampling_Reconstruction_and_Aliasing/01_Introduction_into_Samp/references,chapters/02_Image_Processing_Fundamentals/02_Image_Enhancement_in_the_Frequency_Domain/03_Sampling_Reconstruction_and_Aliasing/02_Impulse_train_Sampling/references,chapters/02_Image_Processing_Fundamentals/02_Image_Enhancement_in_the_Frequency_Domain/03_Sampling_Reconstruction_and_Aliasing/03_Aliasing_and_Anti-alia/references,chapters/02_Image_Processing_Fundamentals/03_Image_Segmentation/01_Basic_Concepts/01_Introduction_Into_Basi/references,chapters/02_Image_Processing_Fundamentals/03_Image_Segmentation/02_Boundary_Based_Methods_and_the_Hough_Transform_/01_Edges_versus_Boundarie/references,chapters/02_Image_Processing_Fundamentals/03_Image_Segmentation/02_Boundary_Based_Methods_and_the_Hough_Transform_/02_Canny_Edge_Detector/references,chapters/02_Image_Processing_Fundamentals/03_Image_Segmentation/02_Boundary_Based_Methods_and_the_Hough_Transform_/03_Hough_Transform_/01_Hough_Transform_for_Lo/references,chapters/02_Image_Processing_Fundamentals/03_Image_Segmentation/02_Boundary_Based_Methods_and_the_Hough_Transform_/03_Hough_Transform_/02_Hough_Transform_for_Lo/references,chapters/02_Image_Processing_Fundamentals/03_Image_Segmentation/02_Boundary_Based_Methods_and_the_Hough_Transform_/03_Hough_Transform_/03_Generalized_Hough_Tran/references,chapters/02_Image_Processing_Fundamentals/03_Image_Segmentation/03_Region_Based_Methods/01_Thresholding/01_Basic_Global_Threshold/references,chapters/02_Image_Processing_Fundamentals/03_Image_Segmentation/03_Region_Based_Methods/01_Thresholding/02_Optimal_Global_Thresho/references,chapters/02_Image_Processing_Fundamentals/03_Image_Segmentation/03_Region_Based_Methods/01_Thresholding/03_Multiple_and_Local_Thr/references,chapters/02_Image_Processing_Fundamentals/03_Image_Segmentation/03_Region_Based_Methods/02_Split-and-Merge_Segmen/references,chapters/02_Image_Processing_Fundamentals/03_Image_Segmentation/03_Region_Based_Methods/03_Region_Growing/references,chapters/02_Image_Processing_Fundamentals/04_Mathematical_Morphology/01_Basic_Concepts/01_Binary_Images_and_Set_/references,chapters/02_Image_Processing_Fundamentals/04_Mathematical_Morphology/01_Basic_Concepts/02_The_Structuring_Elemen/references,chapters/02_Image_Processing_Fundamentals/04_Mathematical_Morphology/02_Basic_Morphological_Operations/01_Erosion/references,chapters/02_Image_Processing_Fundamentals/04_Mathematical_Morphology/02_Basic_Morphological_Operations/02_Dilation/references,chapters/02_Image_Processing_Fundamentals/04_Mathematical_Morphology/02_Basic_Morphological_Operations/03_Opening/references,chapters/02_Image_Processing_Fundamentals/04_Mathematical_Morphology/02_Basic_Morphological_Operations/04_Closing/references,chapters/02_Image_Processing_Fundamentals/04_Mathematical_Morphology/03_Morphological_Algorithms_and_Methods/01_Boundary_Extraction/references,chapters/02_Image_Processing_Fundamentals/04_Mathematical_Morphology/03_Morphological_Algorithms_and_Methods/02_Connected_Components/references,chapters/02_Image_Processing_Fundamentals/04_Mathematical_Morphology/03_Morphological_Algorithms_and_Methods/03_Region_filling/references,chapters/02_Image_Processing_Fundamentals/04_Mathematical_Morphology/03_Morphological_Algorithms_and_Methods/04_Morphological_Reconstr/references,chapters/02_Image_Processing_Fundamentals/04_Mathematical_Morphology/03_Morphological_Algorithms_and_Methods/05_Labelling_Connected_Co/references,chapters/02_Image_Processing_Fundamentals/04_Mathematical_Morphology/03_Morphological_Algorithms_and_Methods/06_Hit-and-Miss_Transform/references,chapters/02_Image_Processing_Fundamentals/04_Mathematical_Morphology/03_Morphological_Algorithms_and_Methods/07_Thinning_and_Skeletoni/references,chapters/02_Image_Processing_Fundamentals/04_Mathematical_Morphology/03_Morphological_Algorithms_and_Methods/08_Distance_Transform/references,chapters/02_Image_Processing_Fundamentals/04_Mathematical_Morphology/03_Morphological_Algorithms_and_Methods/09_Watershed_Segmentation/references,chapters/02_Image_Processing_Fundamentals/04_Mathematical_Morphology/04_Grayscale_Morphology/01_Morphology_with_Graysc/references,chapters/02_Image_Processing_Fundamentals/04_Mathematical_Morphology/04_Grayscale_Morphology/02_Erosion/references,chapters/02_Image_Processing_Fundamentals/04_Mathematical_Morphology/04_Grayscale_Morphology/03_Dilation/references,chapters/02_Image_Processing_Fundamentals/04_Mathematical_Morphology/04_Grayscale_Morphology/04_Opening/references,chapters/02_Image_Processing_Fundamentals/04_Mathematical_Morphology/04_Grayscale_Morphology/05_Closing/references,chapters/02_Image_Processing_Fundamentals/04_Mathematical_Morphology/04_Grayscale_Morphology/06_Top_Hat_Transform/references}


\end{document}